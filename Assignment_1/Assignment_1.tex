\let\negmedspace\undefined
\let\negthickspace\undefined
\documentclass[journal,12pt,onecolumn]{IEEEtran}
\usepackage{cite}
\usepackage{amsmath,amssymb,amsfonts,amsthm}
\usepackage{algorithmic}
\usepackage{txfonts}
\usepackage{mathtools}
\usepackage{gensymb}
\usepackage{amssymb}
\usepackage{tabularx}
\usepackage{multirow}

\DeclareMathOperator*{\Res}{Res}

\begin{document}

\newtheorem{problem}{Problem}
\newtheorem{definition}[problem]{Definition}
\newcommand{\BEQA}{\begin{eqnarray}}
\newcommand{\EEQA}{\end{eqnarray}}

\bibliographystyle{IEEEtran}

\providecommand{\mbf}{\mathbf}
\providecommand{\pr}[1]{\ensuremath{\Pr\left(#1\right)}}
\providecommand{\qfunc}[1]{\ensuremath{Q\left(#1\right)}}
\providecommand{\sbrak}[1]{\ensuremath{{}\left[#1\right]}}
\providecommand{\lsbrak}[1]{\ensuremath{{}\left[#1\right.}}
\providecommand{\rsbrak}[1]{\ensuremath{{}\left.#1\right]}}
\providecommand{\brak}[1]{\ensuremath{\left(#1\right)}}
\providecommand{\lbrak}[1]{\ensuremath{\left(#1\right.}}
\providecommand{\rbrak}[1]{\ensuremath{\left.#1\right)}}
\providecommand{\cbrak}[1]{\ensuremath{\left\{#1\right\}}}
\providecommand{\lcbrak}[1]{\ensuremath{\left\{#1\right.}}
\providecommand{\rcbrak}[1]{\ensuremath{\left.#1\right\}}}

\theoremstyle{remark}
\newtheorem{rem}{Remark}


\newcommand{\sgn}{\mathop{\mathrm{sgn}}}
\providecommand{\abs}[1]{\left\vert#1\right\vert}
\providecommand{\res}[1]{\Res\displaylimits_{#1}} 
\providecommand{\norm}[1]{\left\lVert#1\right\rVert}
%\providecommand{\norm}[1]{\lVert#1\rVert}
\providecommand{\mtx}[1]{\mathbf{#1}}
\providecommand{\mean}[1]{E\left[ #1 \right]}
\providecommand{\fourier}{\overset{\mathcal{F}}{ \rightleftharpoons}}
%\providecommand{\hilbert}{\overset{\mathcal{H}}{ \rightleftharpoons}}
\providecommand{\system}{\overset{\mathcal{H}}{ \longleftrightarrow}}
	%\newcommand{\solution}[2]{\textbf{Solution:}{#1}}
\newcommand{\solution}{\noindent \textbf{Solution: }}
\newcommand{\cosec}{\,\text{cosec}\,}
\providecommand{\dec}[2]{\ensuremath{\overset{#1}{\underset{#2}{\gtrless}}}}
\newcommand{\myvec}[1]{\ensuremath{\begin{pmatrix}#1\end{pmatrix}}}
\newcommand{\mydet}[1]{\ensuremath{\begin{vmatrix}#1\end{vmatrix}}}

\let\vec\mathbf

\vspace{3cm}

\title{
Assignment 1\\AI1110 : Probability and Random Variables
}
\author{Tumarada Padmaja\\CS22BTECH11059}

% make the title area
\maketitle
%\newpage
\bigskip
\textbf{Question:12.13.3.2:} A bag contains 4 red and 4 black balls, another bag contains 2 red and 6 black balls. One of the two bags is selected at random and a ball is drawn from the bag which is found to be red. Find the probability that the ball is drawn from the first bag.
\\
\\
\textbf{Solution:}
\begin{table}[htbp]
\centering
\caption{Random Variable Declaration}
%%%%%%%%%%%%%%%%%%%%%%%%%%%%%%%%%%%%%%%%%%%%%%%%%%%%%%%%%%%%%%%%%%%%%%
%%                                                                  %%
%%  This is a LaTeX2e table fragment exported from Gnumeric.        %%
%%                                                                  %%
%%%%%%%%%%%%%%%%%%%%%%%%%%%%%%%%%%%%%%%%%%%%%%%%%%%%%%%%%%%%%%%%%%%%%%
\begin{tabular}{|c|c|c|}
\hline
Random Variable	&Value	&Event\\
\hline
\multirow{2}{*}X	&0	&not choosing black ball\\
\cline{2-3}
	&1	&choosing black ball\\
\hline
\end{tabular}

\label{table:1}

\end{table}
\bigskip
\\
Given,
\begin{align}
\pr{R=1|B=0}&=\frac{4}{8}=\frac{1}{2} \\ 
\pr{R=1|B=1}&=\frac{2}{8}=\frac{1}{4} \\ 
\pr{B=0}&=\frac{1}{2} \\ 
\pr{B=1}&=\frac{1}{2} 
\end{align}
\pr{B=0|R=1}= probability of choosing bag 1 given that the ball is red \\
From Bayer's theorem,
%\begin{equation}
\begin{align}
\pr{B=0|R=1} &= \frac{\pr{R=1|B=0} \pr{B=0}}{\pr{R=1|B=0}  \pr{B=0} + \pr{R=1|B=1}  \pr{B=1}} \\
&= \frac{\frac{1}{2} \times \frac{1}{2}}{\frac{1}{2} \times \frac{1}{2} + \frac{1}{4} \times \frac{1}{2}}\\
&= \frac{\frac{1}{4}}{\frac{1}{4} + \frac{1}{8}} \\
&= \frac{2}{3} 
\end{align}
%\end{equation}
\\Hence, $$ \pr{B=1|R=1}=\frac{2}{3}$$ 
The probability that the ball is drawn from the first bag is $\frac{2}{3}$ .


\end{document}
